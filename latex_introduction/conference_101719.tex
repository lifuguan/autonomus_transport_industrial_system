\documentclass[cs4size,a4paper]{ctexart}   
%===数学符号公式===
\usepackage{amsmath}    					% AMS LaTeX宏包
\usepackage[style=1]{mdframed}
\usepackage{amsthm}
\usepackage{amssymb}
\usepackage{bm}                      	% 数学公式中的黑斜体
\usepackage{bbm}
\usepackage{amsfonts}
\usepackage{mathrsfs}                	% 英文花体字 体
\usepackage{bbding,manfnt}    			% 一些图标,如 \dbend
\usepackage{lettrine}                	% 首字下沉,命令\lettrine
\def\attention{\lettrine[lines=2,lraise=0,nindent=0em]{\large\textdbend\hspace{1mm}}{}}
\usepackage{longtable}
\usepackage{enumerate}
\usepackage[toc,page]{appendix}
\usepackage{geometry}         			% 页边距调整
\geometry{top=3.0cm,bottom=2.7cm,left=2.5cm,right=2.5cm}
\usepackage[colorinlistoftodos,prependcaption,textsize=small]{todonotes}
%===公式按章编号===
\numberwithin{equation}{section}
\numberwithin{table}{section}
\numberwithin{figure}{section}
%===基本格式预置===
\usepackage{fancyhdr}
\pagestyle{fancy}
\fancyhf{}  
\fancyhead[C]{\zihao{5}  \kaishu 大创文献综述}
\fancyfoot[C]{~\zihao{5} \thepage~}
\renewcommand{\headrulewidth}{0.75pt} 
\CTEXsetup[format={\centering\bfseries\zihao{-2}},name={第, 章}]{section}
\CTEXsetup[nameformat={\bfseries\zihao{3}}]{subsection}
\CTEXsetup[nameformat={\bfseries\zihao{4}}]{subsubsection}
%===图形支持宏包===
\usepackage{graphicx}        			% 嵌入png图像
\usepackage{subfigure}
\usepackage{float}
\graphicspath{{figure/}}
\usepackage{color,xcolor}     			% 支持彩色文本、底色、文本框等
\usepackage[colorlinks,linkcolor=blue,anchorcolor=blue,citecolor=blue]{hyperref}
%\usepackage{caption}
\usepackage[ruled,linesnumbered]{algorithm2e}
%\captionsetup{figurewithin=section}
%===源码和流程图===
\usepackage{listings,fontspec}         	% 粘贴源代码
\newfontfamily\consolas{Consolas}
\definecolor{mygreen}{rgb}{0,0.6,0}
\definecolor{mygray}{rgb}{0.5,0.5,0.5}
\definecolor{mymauve}{rgb}{0.58,0,0.82}
%===颜色===
\usepackage{color,xcolor}
\definecolor{dkgreen}{rgb}{0,0.6,0}
\definecolor{gray}{rgb}{0.5,0.5,0.5}
\definecolor{mauve}{rgb}{0.58,0,0.82}
\usepackage{xcolor}
\lstset{ %
%numberstyle=\tiny\monaco,
%numberstyle=\color[RGB]{0,192,192},
%backgroundcolor=\color{white},   		% choose the background color
backgroundcolor=\color[RGB]{245,245,244},
%backgroundcolor=\color[rgb]{1,1,0.76},
basicstyle=\footnotesize\consolas,       % size of fonts used for the code
identifierstyle=\footnotesize\consolas, 
columns=fullflexible,
breaklines=true,                 		% automatic line breaking only at whitespace
captionpos=b,                    		% sets the caption-position to bottom
tabsize=2,
commentstyle=\color{mygreen}\consolas,   % comment style
%commentstyle=\it\color[RGB]{0,96,96},
escapeinside={\%*}{*)},          		% if you want to add LaTeX within your code
keywordstyle=\color{blue}\consolas,      % keyword style
stringstyle=\color{mymauve}\consolas,    % string literal style
%stringstyle=\rmfamily\slshape\color[RGB]{128,0,0},
frame=single,
rulesepcolor=\color{red!20!green!20!blue!20},
%identifierstyle=\color{red},
language=c++,
framexleftmargin=1.9mm,
xleftmargin=0.4em,
frame=none,
showstringspaces=false,
numbers=none,
}

%--------------------
\hypersetup{hidelinks}
\usepackage{booktabs}  
\usepackage{shorttoc}
\usepackage{tabu,tikz}
\usepackage{float}
\usepackage{multirow}

\tabcolsep=1ex
\tabulinesep=\tabcolsep
\newlength\tikzboxwidth
\newlength\tikzboxheight
\newcommand\tikzbox[1]{%
        \settowidth\tikzboxwidth{#1}%
        \settoheight\tikzboxheight{#1}%
        \begin{tikzpicture}
        \path[use as bounding box]
                (-0.5\tikzboxwidth,-0.5\tikzboxheight)rectangle
                (0.5\tikzboxwidth,0.5\tikzboxheight);
        \node[inner sep=\tabcolsep+0.5\arrayrulewidth,line width=0.5mm,draw=black]
                at(0,0){#1};
        \end{tikzpicture}%
        }
\makeatletter
\def\hlinew#1{%
  \noalign{\ifnum0=`}\fi\hrule \@height #1 \futurelet
   \reserved@a\@xhline}
   
\usepackage{CJK}
\usepackage{ifthen}
\newcommand{\HRule}{\rule{\linewidth}{0.5mm}}
\newcommand{\tabincell}[2]{\begin{tabular}{@{}#1@{}}#2\end{tabular}}%
%===使得公式随章节自动编号===
\makeatletter
\@addtoreset{equation}{section}
\makeatother
\renewcommand{\theequation}{\arabic{section}.\arabic{equation}}
%-------------------------
\usepackage{pythonhighlight}
\usepackage{tikz}                    
\usepackage{tikz-3dplot}
\usetikzlibrary{shapes,arrows,positioning}
%===正文开始===
\begin{document}
%===定理类环境定义===
\newtheorem{example}{例}              	% 整体编号
\newtheorem{algorithem}{算法}	
\newtheorem{theorem}{定理}            	% 按section编号
\newtheorem{definition}{定义}
\newtheorem{axiom}{公理}
\newtheorem{property}{性质}
\newtheorem{proposition}{命题}
\newtheorem{lemma}{引理}
\newtheorem{corollary}{推论}
\newtheorem{remark}{注解}
\newtheorem{condition}{条件}
\newtheorem{conclusion}{结论}
\newtheorem{assumption}{假设}
%===重定义===
\renewcommand{\contentsname}{目录}     
\renewcommand{\abstractname}{摘要} 
\renewcommand{\refname}{参考文献}     
\renewcommand{\indexname}{索引}
\renewcommand{\figurename}{图}
\renewcommand{\tablename}{表}
\renewcommand{\appendixname}{附录}
\renewcommand{\proofname}{证明}
%\renewcommand{\algorithm}{算法} 
\renewcommand\emph[1]{\textcolor{black}{\textbf{#1}}}
%===封皮和前言===
\begin{titlepage}
\begin{center}
% Upper part of the page
\includegraphics[width=0.30\textwidth]{logo}\\[1cm]    
\textsc{\Large Beijing University of Chemical Technology}\\[1.0cm]
\textsc{\Large Computing Methods}\\[0.5cm]
% Title
\HRule \\[0.8cm]
{\huge \bfseries 大创文献综述}\\[0.4cm]
\HRule \\[0.7cm]
% Author
\textsc{李昊,刘本佳,庄梓博,贺怀宇,王昕凯}
\tableofcontents 
\vfill
% Bottom of the page
{创建日期:2020年3月25日}\\
{更新日期:\today}
\end{center}
\end{titlepage}
\pagestyle{plain}
\pagenumbering{Roman}
\thispagestyle{empty}
%===正文===
\pagestyle{fancy}
\pagenumbering{arabic}

\abstract
不需要对整个场地进行定制(地面标识,摩擦系数),使其拥有特别高的快速部署和适应能力(对场地进行建模即可投入运行)
%===第一章===
\section{AGV发展现状}

\subsection{研究进展}

\subsection{公司产业落地情况}

%===第二章===
\section{工程分解:硬件部分}

\subsection{AGV底盘}

\subsection{LiDAR传感器}

\subsection{RGBD传感器}

%===第三章===
\section{工程分解:软件部分}
\subsection{SLAM算法描述}

\subsection{路径规划算法描述}

\subsection{AGV姿态控制算法描述}

\subsection{基于RGBD的物体识别与定位算法描述}

\subsection{基于Web和MySQL的控制终端描述}

\section{实验部分}

\subsection{GUI控制程序演示}

\subsection{仿真模型构建}

\subsection{流程图}

\subsection{运行情况}

%===参考文献===
%\addcontentsline{toc}{section}{参考文献}
%\bibliographystyle{abbrv}     %论文引用格式
%\bibliography{E:/studio/write/bibkit/wholebiblio}
                         
\begin{thebibliography}{99}
\bibitem{A19}
Author. {\em \color{red}Title}. \url{http://www.baidu.com}, 2020.

\bibitem{B19}
Author. {\em \color{blue}Title}. \url{http://www.baidu.com}, 2020.

\bibitem{C19}
Author. {\em Title}. \url{http://www.baidu.com}, 2020.

\bibitem{D19}

\end{thebibliography}
\end{document}
%===结束===



